\documentclass[margin,line]{res}
\usepackage{url}
\usepackage[usenames, dvipsnames]{color}
\usepackage[hidelinks]{hyperref}
\oddsidemargin -.65in
\evensidemargin -.65in
\textwidth=6.0in
\textheight=8.5in
\itemsep=0in
\parsep=0in
\topmargin= 0in
\topskip=0in
\newenvironment{list1}{
  \begin{list}{\ding{113}}{%
      \setlength{\itemsep}{0in}
      \setlength{\parsep}{0in} \setlength{\parskip}{0in}
      \setlength{\topsep}{0in} \setlength{\partopsep}{0in}
      \setlength{\leftmargin}{1in}}}{\end{list}}
\newenvironment{list2}{
  \begin{list}{$\bullet$}{%
      \setlength{\itemsep}{0in}
      \setlength{\parsep}{0in} \setlength{\parskip}{0in}
      \setlength{\topsep}{0in} \setlength{\partopsep}{0in}
      \setlength{\leftmargin}{0.2in}}}{\end{list}}
 \tolerance=1
 \emergencystretch=\maxdimen
 \hyphenpenalty=10000
 \hbadness=10000
%\newsectionwidth{1in}
\begin{document}
\name{
	{\Large Vignesh Vittal Srinivasaragavan - Curriculum Vitae \quad \quad}
}

\begin{resume}
\section{\sc Current\\ Address}
\begin{tabular}{@{}p{3.8in}p{3in}}
198 Hoosick Street,            & {Phone:} +1-(518)9618823 \\
Apt \#3, & {E-mail:} \href{mailto:vickyragav95@gmail.com}{\color {RoyalPurple} vickyragav95@gmail.com}\\
Troy, NY - 12180 & {Website: }{\href {https://goo.gl/3anJQh} {\color {RoyalPurple} https://goo.gl/3anJQh}}\\
\end{tabular}

\section{\sc Profile \\Summary}
	An industrious engineer with a strong research background in {\bf computational mechanics/mathematics} complimented by {\bf programming skills}. Currently pursuing a doctoral degree with research focus on {\bf adaptive numerical methods for stochastic PDEs} 
\section{\sc Education}
{\bf Rensselaer Polytechnic Institute}, New York, USA\hfill {\em (Aug '17 -- present)}\\
PhD in Mechanical Engineering \hfill{\em GPA 3.83/4.00}\\
{\em Advisor: \href {http://homepages.rpi.edu/~sahni/} {\color{RoyalPurple} Dr. Onkar Sahni}}\\\\
{\bf Indian Institute of Technology Madras}, India \hfill {\em (Aug '12 -- Jul '17)}\\
B,Tech/M.Tech. in Mechanical Engineering (Product Design) \hfill{\em GPA 8.38/10.00}\\
{\em Minor: Industrial Engineering} \hfill {\em Major GPA 8.61/10.00}

%{\bf Maharishi International Residential School}, India \hfill {\em (Aug '08 -- Jul '12)}\\
%Senior Secondary Certificate (Class XII) \hfill{\em Percentage 89.8 \%}\\
%{\em Elective Subject: Computer Science} \hfill {\em Top 1 \%}

\section {\sc Research Interests}
\textbullet\hspace{0.005cm} {\bf Uncertainty quantification:} Stochastic PDEs, Intrusive and Non-Intrusive UQ, Stochastic Finite Elements, Multi-level and Multi-fidelity UQ\\
\textbullet\hspace{0.005cm} {\bf Fluid Mechanics:} Computational Fluid Dynamics, Stabilized Finite Elements

\section {\sc \href{https://sites.google.com/site/vigneshsrinivasaragavan/research}{\color{RoyalPurple}Research Projects}}
%\textbullet\hspace{0.005cm} \href{}{{\bf Stochastic Variational Multi-scale methods}}\hfill {\em (Sep '17 -- Dec '17)}\\
%o {\em Done as a part of the course ``Fundamentals of Finite Elements" taught by Dr Mark Shephard}\\
%o {\em Skills and Tools:  MATLAB}\\
%- Performed a detailed literature survey of research papers on the topics Variational multiscale methods, Stochastic computations and Stochastic VMS\\
%- Developed custom MATLAB codes for stochastic VMS and verified its working for a 1D stochastic Advection-Diffusion problem\\
%- Extended the codes to incorporate non-linear stochastic burger's equation problem

\textbullet\hspace{0.005cm} \href{https://sites.google.com/site/vigneshsrinivasaragavan/research}{{\bf Wavelet methods for Linear-Elastic Solids}}\hfill {\em (Aug '16 -- May '17)}\\
o {\em Guide: \href {https://mech.iitm.ac.in/meiitm/personnal/raju-sethuraman/} {\color{RoyalPurple} Dr. Raju Sethuraman}, Computational Mechanics Lab, Machine Design Section, IIT Madras}\\
o {\em Skills and Tools:  Mathematical modeling, MATLAB}\\
- Conducted a detailed research on solving ODE/PDEs using Haar wavelets with an emphasis on implementing the same for Linear Elastic equations\\
- Developed custom MATLAB codes for the same and achieved greater precision, convergence while minimizing computations\\ 
- Performed extensive research on Wavelet-Galerkin methods for ODE/PDEs and developed MATLAB codes that evaluates necessary functions (like wavelet integrals, moment terms and connection coefficients) associated with the method\\
%- Currently developing Wavelet-Galerkin algorithms to solve Linear-Elastic differential equations\\
- Thrust also placed on investigation of effect of parameters (like genus of wavelet and resolution used) on convergence and stability of the solutions

\textbullet\hspace{0.005cm} \href{https://sites.google.com/site/vigneshsrinivasaragavan/research}{{\bf Modeling, Simulation and Control of a Robot }}\hfill {\em (Dec '14 -- Aug '16)}\\
o {\em Guide: \href {https://home.iitm.ac.in/sspandian/} {\color{RoyalPurple} Dr. S. Soundarapandian}, Manufacturing Engineering Section, IIT Madras}\\
o {\em Skills and Tools: 3D modeling, Structural analysis, Kinematic and Dynamic analysis, Control systems, ADAMS, MATLAB, SIMULINK, SolidWorks}\\
- Reverse engineered a robot, created a 3-D model and constructed a path planning algorithm for the same \\
- Designed the control software for the manipulator arm, which ensures precise and accurate path adherence in minimally-invasive orthopedic surgery applications \\
- Validated the model and the control systems by creating a co-simulation in the ADAMS environment, with custom MATLAB codes in a SIMULINK module \\
- An aspect of the project was presented in the {\em 3$^{rd}$ International Conference on Mechatronics and Mechanical Engineering} held at Shanghai, October 2016 and published in the conference proceedings

\section {\sc Academic Experience}
\textbullet\hspace{0.005cm} {\bf Teaching Assistant, Rensselaer Polytechnic Institute}\hfill {\em (Aug '17 -- Dec'17)}\\
o {\em Course: Engineering Dynamics / Course instructor: \href {} {\color{RoyalPurple} Dr. Jeremy Laflin}}\\
- Supervised a class of $\sim$45 undergraduate students in the sophmore level course\\
- Assisted course instructor in class, assignments and proctoring examinations

\textbullet\hspace{0.005cm} {\bf Teaching Assistant, Indian Institute of Technology Madras}\hfill {\em (Aug '16 -- Nov'16)}\\
o {\em Course: Advanced Mechanics of Solids / Course instructor: \href {https://mech.iitm.ac.in/meiitm/personnal/raju-sethuraman/} {\color{RoyalPurple} Dr. Raju Sethuraman}}\\
- Supervised $\sim$60 undergraduate and post graduate students in the advanced-level course\\
- Assisted course instructor in conducting class tutorials and examinations

\section {\sc Industrial Experience}
\textbullet\hspace{0.005cm} {\bf Winter Intern, Forbes Marshall Ltd. }\hfill {\em (Dec '15 -- Jan'16)}\\
- Mathematically modeled the concentration factor of a Fresnal-type Evacuated Tube Collector in terms of input design parameters\\
- Estimated the optimal parameter set by running a Monte Carlo simulation

\textbullet\hspace{0.005cm} {\bf Summer Intern, GE India Pvt. Ltd, Transportation division}\hfill {\em (May '15 -- Jul'15)}\\
- Optimized the parameters in Variable Valve Timing (VVT) mechanism in GE Engines\\
- Suggested possible noise mitigation and heat screening methods to be implemented in GE Engines\\
- Generated a Requirement Traceability Matrix (RTM) for Lube Oil pump test rig

\section {\sc Skills}
\textbullet\hspace{0.005cm} {\bf Modeling} : CreO Parametric, AutoCAD, Inventor, SolidWorks \\
\textbullet\hspace{0.005cm} {\bf Analysis and Simulation} : MATLAB, SIMULINK, Paraview, Adams, ANSYS, C/C++\\
\textbullet\hspace{0.005cm} {\bf Symbolic Computation} : Maple, Mathematica\\
\textbullet\hspace{0.005cm} {\bf Presentation and Documentation} : \LaTeX 

\section{\sc Scholastic Achievements}
\textbullet\hspace{0.005cm} Ranked in {\bf top 1\%} in the \href{https://en.wikipedia.org/wiki/Indian_Institute_of_Technology_Joint_Entrance_Examination}{\bf IIT-JEE 2012} (from over 0.5 million applicants)\\
\textbullet\hspace{0.005cm} Ranked in {\bf top 1\%} in the \href{https://en.wikipedia.org/wiki/All_India_Engineering_Entrance_Examination}{\bf AIEEE 2012} (from over 1.2 million applicants)\\
\textbullet\hspace{0.005cm} Qualified for the {\bf Indian National Maths Olympiad} 2011 (Among the {\bf top 500} in India)\\
\textbullet\hspace{0.005cm} Secured {\bf top 1\%} in state in {\bf National Standard Examination in Physics} 2011

\section {\sc Publications}
\textbullet\hspace{0.005cm} ADAMS-MATLAB Co-Simulation of A Serial Manipulator, Tejaswin  Parthasarathy, Vignesh Srinivasaragavan, Soundarapandian  Santhanakrishnan. MATEC Web Conf. 95 08002 (2017) DOI: 10.1051/matecconf/20179508002
\end{resume}
\end{document}	